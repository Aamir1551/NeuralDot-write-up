\section{Current Systems}
Due to the current AI explosion, AI libraries have become very common, with the leading library being Tensorflow made by Google's brain team for their machine learning research. However, Tensorflow isn't the only library out there, others include theano, pytorch, openNN and many others. However, many of these libraries are made for different purposes and weren't made for beginners in mind and certainly not for A level students with limited knowledge of AI.

\subsection{Tensorflow}
Currently, Tensorflow is the leading AI library and has been made especially for researchers and for industrial use. Tensorflow, being one of the best libraries for machine learning in general however, requires extensive knowledge from linear algebra to optimisation. This makes it very limiting for many student already as many of these topics are not studied at A level and all they would want is an easy to use machine learning library that allows them to experiment with their ML models. 

In addition, Tensorflow also lacks in its user-friendliness, as making a simple Neural Network would require a lot of work. This is because Tensorflow's main purpose isn't for NNs but for allowing the user to create a computational graph that they can experiment with. This means that when creating a simple neural network, the user would need to set up the dimensions of the matrices and the biases being used, when in fact all that was required was a single number representing the number of neurons in that layer. After this the user would also need to mathematically define the cost function being used and also create some placeholders for the training set and the test set. All this makes it very difficult for the user especially if the user isn't a confident programmer who wants to experiment with simple NNs.

\subsection{Pytorch and Theano}
Pytorch and Theano, are also one of leading AI libraries that are being used. However, Pytorch and Theano both face the same problem of being overly complicated for beginners as they are more focused on intermediate AI engineers and research teams. Furthermore, Pytorch and Theano are only restricted to python which again puts a barrier for VB users as they also need to learn a new language, thus extending the barrier for many beginners. 

\section{Target Audience and Clients}
My target audience are A level students that are passionate about AI, but have a very limited amount of knowledge regarding AI. Therefore, I have interviewed many students and asked for their opinion regarding my initial design of the library.
\bigbreak
One student, Nitish Bala, I interviewed said \textit{"The library should be easy to use by making sure many of the parameters such as the initial weights should already be defined and creating a NN shouldn't require someone to know all the maths behind it. The library should also make it easy for users to add convolutional layers and define their own layers which can also be trained by a pre-defined gradient descent optimiser."}
\bigbreak
Another student, Basim Khajwal, said \textit{"The library should allow the user to create their own back propagation algorithms and also allow the user to experiment with different models. The library should be easy to use by limiting the amount of setting up required by the user such as matrix sizes and defining cost functions from definition. Finally, users should be allowed to create many NN at the same time to compare the performance of one model to another."}
\bigbreak
A third student, Taha Rind, said \textit{"The library shouldn't be too complicated and that experimenting with different parameters should be easy to do. Adding a dense layer should at minimum require the user to input the number of neurons in a layer and layer activation and there should be some advanced optimisation algorithms to train these dense networks such as momentum. Finally, the library should also allow the user to see the parameters learnt and the gradients of the dense layers in any layer."}
\bigbreak
A fourth student, Mujahid Mamaniat, I interviewed said \textit{"The library must offer easy manipulation of matrices by making sure many of the functions required are already implemented such as rotation, reshaping and adding/removing a column in a matrix. The library must also further allow the user to manipulate with images in RBG format, which may be done through the use of lists of matrices. Finally, making a neural network should require little effort and should not be difficult to make."}
\bigbreak
Finally, a fifth student, Jamie Stirling , I interviewed said
\textit{"The library should have extra focus on dense-nets as many beginners do not understand how conv-layers work. I remember when I tried experimenting with my first ML library. It was difficult to use as it supported many different types of layers which over complicated it, as I didn't understand most of what it offered, as all I knew about was dense nets. Therefore, making a dense net should be incredibly easy as that is all many beginners in AI know about. One way in which it could be made easy would be by setting many of the parameters optional such as the initial weights. By having many optional parameters, I think the user would worry less about whether they have implemented the net correctly."}
\bigbreak
From these interviews it is clear that my target audience are looking for an easy to use machine learning library that allows them to experiment easily with different configurations of Neural Networks. I will be focusing more on dense-nets as it is clear that many beginners do not have a wide knowledge regarding NNs. Therefore, I will try to make it as easy as possible to make a dense net, and also add some extra functionality to dense nets, making it more functional for the user to work with. 
These extra functionality will include allowing the user to view the gradients of the dense-layers, thus enabling users to view the change in gradients as the network learns. This will allow the user to develop an insight into NNs, thus easing the way for beginners in AI.
Furthermore, I will also be including conv-layers as many students will quickly learn the basics of NNs and would want to move on to complex data such as images or sound. Therefore, including conv-layers will allow users to experiment with all kinds of data. However, my main focus will be on dense nets as many beginners would not have the necessary skills to use conv nets and would just want to experiment with the dense-layers due to their limited knowledge.
Finally, I will also be interviewing while the making of the project and keep asking users as to which parts of library can still be improved further.


