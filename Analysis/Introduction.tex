\section{Introduction}
Machine Learning has recently become a very popular field in Computer Science thanks to an area known as "Deep Learning", which has allowed researches to achieve record breaking results in tasks such as computer vision, natural language processing, drug synthesis, health care, financial trading, physical simulation and many other tasks, which were difficult to do before the rise of Deep learning. Thus there are many libraries online for researchers, AI scientists or students with the relevant knowledge to carry out their tasks, however for students that are passionate and have very limited knowledge of AI, it can be a daunting experience to learn how to use a machine learning library.

The hardest part however, is the theory behind the machine learning which is required to make full use of the library, can be extremely difficult as the mathematics behind the algorithms require graduate level mathematics, causing many students to give up, even before building their very own first ML model.

\section{Application of AI/ML libraries}
ML libraries, are there to make the process of creating an AI for a specific task seamlessly easy. The core algorithms running under the surface for these libraries, include matrix operations which includes inverse, determinant, multiplication, back propagation with its many variations to reduce a cost function, convolutions and many other heavy computational tasks. By using these libraries, the user can focus on the important tasks such as choosing the best parameters for their ML models. Once the user has made their ML model, they can then use it for their task. 
\\ \\
Some examples are:
\begin{itemize}
    \item Using a ML model for a bio-metric recognition system, such as face, sound or some other feature
    \item Using a ML model for a game playing agent, this could include making an AI opponent for a game of noughts and crosses to making an AI to play mine-craft.
    \item Using a ML model to simulate biological systems over time, this may include natural selection or even see how animals may compete against each other and how changes in the initial parameters such as the size, colour or any other parameter may have an impact on the system over time 
    \item A ML model may be used to make accurate predictions about some chaotic system. This may include predicting the weather, stock markets or even immigration patters.
    \item A ML model may be used to develop a self driving vehicle, which can include cars, aeroplanes and ships.
\end{itemize}
The most impressive part about this is that developing an AI model for a self driving vehicle wouldn't be much different to developing an AI model for a simulation. This is because the basic framework behind the model is the same, while the only difference being are the parameters chosen such as the model architecture, size and the training data, but the principles remain the same. This just shows how easy the process becomes using a machine learning library. 

From this we can see that the applications of AI are endless, hence it is no surprise that many beginners would not want to miss out on any opportunity with the on-going hype of AI.

\section{Machine Learning Techniques}
All machine learning models work the same way, with the only difference being the inference process. Every machine learning model has 2 different phases. The training phase and the inference phase.

Before progressing to the training phase, a machine learning engineer should have a training set. A training set is split up into 2 difference sets. One set is used to make the predictions while the others are used as labels for the other set. 
An example of this is for a home face recognition. Say, you wanted to make a ML model that can detect faces inside the house. So the training data would in this case be the face images of the people that live in the house and the labels would be the name of each face in the training set.
\\ \\
The training phase is as follows:
\begin{enumerate}
    \item An inference is made using the training data
    \item The outputs of the inference is then compared with the labels of the training set to find the error for each training iterations using a $cost function$.
    \item The parameters of the ML model are updated using a back propagation algorithm
\end{enumerate}

This training phase is then repeated many times until the error is less than the acceptable bound that was set.

\section{Neural Networks} 
Neural Networks by definition are general purpose function approximators. Neural Networks are currently one of the best machine learning approaches to take, mainly due to the hardware available and numerical computation has now became very cheap. Being a breakthrough, NNs have become the hottest field in AI. Due to this, many beginners and enthusiast are learning how to develop their own NN for their application, however many of the libraries developed require a strong understanding behind NN and the mathematics involved.