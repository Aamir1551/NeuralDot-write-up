\section{Methods of Research}
I will be researching in several ways which ranges from online courses to articles. I will also be comparing my product against other existing products in the market and see how it compares. I will also be experimenting with other machine learning libraries to gain a strong ground on how current machine learning libraries are and comparing their strengths and weaknesses.

\subsection{Coursera}
I will be using Coursera as it will allow me to gain a better and deeper understanding in machine learning. Currently, I am taking a course in deep learning which will give me the foundations and skills required to understand neural networks and also expose me to the other different types of variations of NN such as convolutional networks, recurrent neural networks and LTSM networks. By being exposed to these new algorithms, I will be able to improve the functionality of my library making it more flexible to use, as there are constant advancements being made in the field of AI. Thus by making the library more flexible the user wouldn't feel restricted in experimenting with new ideas, this could include, a different cost function, activation function or even a different layer.

\subsection{Khan Academy}
I will be using Khan Academy to learn linear algebra and multi-variable calculus for my neural network course. This is because there is a lot of linear algebra involved in machine learning such as tensor products and matrix calculus. Therefore by taking a course in linear algebra, I will be able to deepen my knowledge and develop an insight into how the algorithms should be implemented. 

\subsection{Medium}
I will also be using medium as a method of research, as there have been many articles on medium that explain the different machine learning algorithms. I will be using Medium as it will expose me to the different machine learning algorithms and approaches people have taken. This will help me to learn from other people's techniques and the approaches they took when encountering a problem.