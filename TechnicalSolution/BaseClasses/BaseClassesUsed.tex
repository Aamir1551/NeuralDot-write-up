\section{Base Classes Used}
\begin{table}[H]
\centering
    \begin{tabular}{|p{3cm}|p{13cm}|}
    \hline
    Base Class Implemented & Reason for Implementation \\ \cline{1-2}
    \textbf{Tensor} & The class \textbf{Tensor} is an \textit{abstract class} as it is never instantiated. The class Tensor is \textit{inherited} by the classes \textbf{Volume} and \textbf{Matrix} as both share functions and sub-routines in common. In addition to this, the class \textbf{Tensor} is an abstract class as the class \textbf{Layer} is a \textit{generic class} of type \textbf{Tensor}, allowing its derived classes to be of derived classes of type \textbf{Tensor}. Furthermore, functions in other classes such as \textbf{netData} also use both matrices and volumes implicitly, therefore, it is necessary to have a base class that both \textbf{Volume} and \textbf{Matrix} can inherit from.  \\ \cline{1-2}
    \textbf{Layer} & The class \textbf{Layer}, unlike \textbf{Optimiser} and \textbf{Tensor} is an \textit{interface} meaning no functions and subroutines are defined but instead gives a list of functions or sub-routines that must be defined in the classes that \textit{extend} from it. This interface is then extended by the \textbf{Dense}, \textbf{Conv}, \textbf{Reshape} and \textbf{Maxpool} classes. This saved time, as they had the same functions and therefore the interface \textbf{Layer} provided a template for creating these classes. Furthermore, this interface is a \textit{covariant generic interface}. The reason for this is that the class Net, stores a \textit{stack} of layers defined by the user. This is an example of \textit{composition} that was used in my project. Therefore, if the interface \textbf{Layer}, was not \textit{covariant} the \textit{derived} classes, such as \textbf{Dense}, \textbf{Conv}, \textbf{Reshape} and \textbf{Maxpool} would not be able to be pushed onto the stack as this would result in a run-time error. Therefore, to solve this issue, it was necessary to make the interface Layer a \textit{covariant generic interface} instead. Furthermore, having layer as a \textit{generic interface} makes intuitive sense as well, as the specific parameters for a particular function in a layer will depend upon the type of layer it is, as this will then dictate the type of the output from the function. Finally, one of my key extended objectives was enabling the user to create their own layers. Therefore, by having an interface called \textbf{layer}, the user can easily define their own layers by extending from the \textbf{Layer} interface. This allows easy creation of new layers, without having to change all the other classes just to integrate the new layer defined. \\ \cline{1-2}
    \textbf{Optimiser} & The \textbf{Optimiser} class is a \textit{MustInherit} class that is inherited by \textbf{GradientientDescentOptimiser} and \textbf{AdvancedOptimiser}. This class includes all the necessary functions a back-propagation algorithm may require. This makes it easy for users to create their own back-propagation algorithms, which was one of my extended objectives. Furthermore, the class also includes some key functions such as \textit{splitdata}, \textit{resetParameters}, \textit{calculateCost} and \textit{CalculateGradients}. These functions are used by the derived classes in their back-propagation procedure, by communicating with the base class. Furthermore, the class \textbf{Optimiser} has a \textit{MustOverride} function called \textit{run}, which is where the actual back-propagation procedure occurs and is used to train the net. By having this \textit{abstract class}, the user can easily create their own optimisation algorithms by defining the \textit{run} procedure of their optimisation class. \\ \cline{1-2}
    \textbf{Advanced Optimiser} & This base class was used as the optimisation methods, \textbf{Adam}, \textbf{Momentum} and \textbf{RMS} only work on dense nets. Therefore, it was necessary to have a base class that can avoid these optimisation methods being used to train the conv nets or other user-defined nets. Finally, if the user wanted to create an optimisation algorithm for dense nets, the user could \textit{inherit} from the \textbf{AdvancedOptimiser} class instead of inheriting from the \textbf{Optimiser} class. \\ \cline{1-2}
    \end{tabular}
    \caption{Base Classes Used}
\end{table}

Below is the following code used to create the 3 most important base classes in my project:

\begin{minted}[
frame=lines,
breaklines=true
]{vb.net}

Public Interface Tensor
    'The base class tensor, will be inherited by the Volume and the Matrix class.
    'Also both Volume and Matrix are tensors and will both have functions that are in common.


    Sub print() 'The sub print is used to print out the values of the Tensor. This is nessesary that every child inherits this as the user may want
    'to see all the values the Tensor holds
    Sub transposeSelf() 'This subroutine transposes the Tensor. This is a useful operation as Transpose is used many times in deep-nets, especially for back-prop

    Function clone() As Tensor 'This function will be used by all Tensors, when cloning every layer. This clone function returns the exact same Tensor, with
    'the same values and same state.
    Function normalize(Optional ByVal mean As Double = 0, Optional ByVal std As Double = 1) As Tensor 'This function will be used to normalise the values
    'in a matrix.
    Function getshape() As List(Of Integer) 'Function returns the shape of the Tensor. Function returns a list as the tensors can have an arbitrary number
    'of dimensions.

End Interface

Imports NeuralDot

Public Interface Layer(Of Out T As Tensor)
    'This base class Layer will be inherited by Dense, Conv, MaxPool and Reshape
    'This base class will hold the common function that all thses layers will use
    'This is a generic class of type Tensor, as all layers must be of type Tensor

    ReadOnly Property parameters As List(Of Tensor)
    'The parameter property will be in common for all layers as all layers will need to output the variables they are storing
    'This property is useful for back-prop and debugging

    Function f(ByVal x As Tensor) As T
    'Function will be used to forward propagate through a layer
    Function clone() As Layer(Of T)
    'Function will be used to clone a layer. This is useful when saving a model as all layers will need to be cloned when saving a model
    Function update(ByVal learning_rate As Decimal, ByVal prev_delta As Tensor, ByVal ParamArray param() As Tensor) As Tensor
    'This function is used when a layer depends upon the prevous layers parameters. Therefore, this function is a MustInherit, as when the user defines their
    'own functions they may need to use this depending upon how forward propagation works within the layers

    Function update(ByVal learning_rate As Decimal, ByVal prev_delta As Tensor) As Tensor
    'This function updates the parameters usisng prev_delta which is the gradient of loss function w.r.t the parameters
    'The applicability of this function depends upon how the forward propagation works in this layer.

    Sub deltaUpdate(ByVal ParamArray deltaParams() As Tensor)
    'This sub-routine updates the parameters that are being trained, using deltaParams as the respective gradient via backprop
End Interface

Public MustInherit Class Optimizer
    Public ReadOnly model As Net, dataxy As IEnumerable(Of Tuple(Of Tensor, Tensor))
    Public iterations As Integer = 0, losses As New List(Of Tensor) 'The list losses will store all the losses for every iteration
    'The variable Model stores the Net being trained by referenced. This means when an update occurs the net is updates
    'dataxy stores the data that will be used to train the net

    Public Sub New(ByRef _net As Net, ByVal xydata As IEnumerable(Of Tuple(Of Tensor, Tensor)))
        model = _net
        dataxy = xydata
    End Sub

    MustOverride Function run(ByVal learning_rate As Decimal, ByVal printLoss As Boolean, ByVal batchSize As Integer, ByVal ParamArray param() As Decimal) As List(Of Tensor)
    'This function "run" is MustOverride, as every optimiser must have a method to train the net using a mini-batch.
    'There is no need to hardcode batch or stoachastic gradient descent as they are just special cases of mini-batch gradient descent, i.e batchsize = m, batchsize = 1, respectively.
    'The iterations variable describe the number of training iterations, used to train the net
    'If printLoss = True, then the loss is printed out on each training epoch
    'param() denotes the parameters that will be used by the optimiser in traning the net
    'After each training epoch, the error will be stored in a list, which is returned by this function

    MustOverride Sub resetParameters() 'This sub-routine resets the parameters being used to train the net. This includes the iterations variable.

    Public Function calculateCost(ByVal xydata As IEnumerable(Of Tuple(Of Tensor, Tensor))) As Matrix
        Dim temp As New Matrix(1, 1)
        For j As Integer = 0 To xydata.Count - 1
            temp += model.loss.f(model.predict(xydata(j).Item1), xydata(j).Item2)
        Next
        Return temp / xydata.Count
    End Function
    'This function returns the average cost of the net using the current weights

    Public Function splitdata(ByVal batchSize As Integer) As List(Of IEnumerable(Of Tuple(Of Tensor, Tensor)))
        Dim batchdata As New List(Of IEnumerable(Of Tuple(Of Tensor, Tensor))) 'This list will store all the data for the seperate batches
        For batchNum As Integer = 0 To dataxy.Count / batchSize - 1
            Dim temp As New List(Of Tuple(Of Tensor, Tensor)) 'The Temp list will store the examples for a particular batch for the gradient descent
            For n As Integer = 0 To batchSize - 1
                temp.Add(dataxy(batchNum * batchSize + n))
            Next
            batchdata.Add(temp.AsEnumerable)
        Next
        Return batchdata
    End Function 'This function will organise all the data examples into seperate batches for minibatch gradient descent


    Public Function calculateGradients(ByVal xypoints As IEnumerable(Of Tuple(Of Tensor, Tensor))) As Tuple(Of List(Of Matrix), List(Of Matrix))

        Dim pred As Tensor
        Dim errors As New List(Of Matrix)
        For Each point In xypoints
            pred = model.predict(point.Item1)
            errors.Add(model.loss.d(pred, point.Item2) * model.netLayers.Peek.parameters.Last)
        Next
        Dim deltas As New Stack(Of Tensor)
        deltas.Push((New Volume(errors) / dataxy.Count).mean(2))

        Dim d As IEnumerable(Of Matrix) = DirectCast(model.netLayers(0), Dense).gradient(deltas.Peek)
        Dim dw As New List(Of Matrix)({d(0)})
        Dim db As New List(Of Matrix)({d(1)})
        For layer As Integer = 1 To model.netLayers.Count - 1

            Dim dlayer As IEnumerable(Of Matrix) = DirectCast(model.netLayers(layer), Dense).gradient(deltas.Peek, model.netLayers(layer - 1).parameters(0))
            dw.Add(dlayer(0))
            db.Add(dlayer(1))
            deltas.Push(db(layer))
        Next
        Return New Tuple(Of List(Of Matrix), List(Of Matrix))(dw, db)
    End Function 'This function calculates the gradients for a batch of xypoints, i.e mini-batch gradient descent

End Class

\end{minted}