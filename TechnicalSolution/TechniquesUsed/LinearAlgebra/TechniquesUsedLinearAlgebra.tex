\subsection{Linear Algebra Techniques Used}
\begin{table}[H]
\centering
    \begin{tabular}{|p{2.2cm}|p{6cm}|p{7cm}|}
    \hline
    Technique Used (\& Class Implemented) & How it works & Reason for implementation \\ \cline{1-3}
    
    Matrix Operations (Matrix) & See section 2.1.1.3.4  & See section 2.1.1.3.4 \\ \cline{1-3}
    
    Volume Operations (Volume) & The Volume operations work very similar to the Matrix operations. The only difference between the two is that the Volume operations work on volumes instead. An example of this is the multiplication of two Volumes element-wise. Multiplication element-wise of 2 matrices involve multiplying every element in one matrix with the corresponding element in the other. The same applies for Volumes. The elements in one volume are multiplied element-wise with the elements in the other. This can be extended to other operations such as "-", "+" and "/". & There are two main reasons why these operations were implemented in the \textbf{Volume} class. The first reason is that these operations are used thoroughly in the \textbf{Conv} class for back-propagation and also for the foreword propagation. Furthermore, these operations are also used when handling with Volumes. This is because when the user is dealing with images, sound or some other data format which have a 3rd dimension, it will be much easier for the user to manipulate using the \textbf{Volume} class, as most of the most common data-processing functions will be implemented. This includes \textit{normalising}, \textit{rotating}, \textit{splitting} a portion of the volume and so on. An example of a situation where the user may want to use these operations is when the user is dealing with images in RGB format, the user may want to \textit{normalise} the values for the conv net, therefore by having a normalising function in the \textit{Volume} class makes it much easier for the user as they will not need to define their own functions for the data processing part. Another function that was implemented was \textit{op(f(x), v)}, where f(x) is a function that takes in a matrix and returns a matrix and V is a volume which this function will be applied on layer by layer. This particular function allows many of the matrix operations to be extended to the Volume class.  \\ \cline{1-3}
    
    Matrix Multiplication (Matrix) & Matrix Multiplication works by Multiplying the rows of one matrix with the columns of the the other and summing in the process. For more information about matrix multiplication see section \ref{SMatrixMultiplication}, which is in design. & The reason I implemented matrix multiplication was that it is used many times in dense networks to propagate through the layers. \\ \cline{1-3}
    
    Matrix Iteration (Matrix) & The matrix iteration algorithms work by yielding an item instead of returning a specific item and then terminating. Matrix Iteration for the columns in a matrix works by yielding columns in the matrix which then allows the user to loop over these iterations. The \textit{submatrix} iterator works by striding a window over a matrix and then returning all the elements the window strides over. & Matrix iteration are extremely useful to my library as it saved me from writing the same for-loops many times and thus makes my code much easier to read and understand. These iteration algorithms can also be used by the user as the user may want to loop over their data which may be in matrix format. Furthermore, the iteration \textit{submatrix} is used in the \textbf{Matrix} class for the functions \textit{conv}, \textit{maxpool} and is also used in the MaxPooling class, as it implements the required for-loops these functions require. \\ \cline{1-3}
    \end{tabular}
    \caption{Techniques Used - Linear Algebra (part 1)}
\end{table}

\begin{table}[H]
    \centering
    \begin{tabular}{|p{2.2cm}|p{6cm}|p{7cm}|}
    \hline
    Volume Iterations (Volume) & The iterations for the \textbf{volume} class work similarly to the iterations in the \textbf{Matrix} Class. The \textbf{Volume} class implements the iterators, \textit{subvolume} and \textit{Items}. The subvolume iterator works the same way as the submatrix iterator with the exception that subvolume works with volumes instead. Finally, the iterator Items work by yielding the layers of a Volume.
    & The reason for implementing these iterators are the same for the Matrix Iterators, as they both reduce the need to use the same for-loops to iterate over the same data-structure. \\ \cline{1-3}
    
    Volume Casts (Volume) & Volume casting is used to cast a given to matrix or to cast a matrix to a volume. & These functions are used by the \textbf{Reshape} class, as this layer transforms the output volume from a \textbf{Conv} class to a matrix for the \textbf{Dense} class that follows on. For the back-prop procedure however, the casting is from Matrix to Volume, which will then be back-propagated to the conv layers as the deltas for the back-prop procedure. \\ \cline{1-3}
    
    Convolution (Volume) & See \ref{SConvolution} & Convolution is used to forward propagate and also back-propagate throughout the conv classes as it is part of the inference process for convolutional neural networks. \\ \cline{1-3}
    \end{tabular}
    \caption{Techniques Used - Linear Algebra (Part 2) }
\end{table}

Below is the code for some of the matrix operations that are being used in my project: 
\begin{minted}[
frame=lines,
breaklines=true
]{vb.net}
    Public Function reshape(ByVal rows As Integer, ByVal cols As Integer) As Matrix
        If rows * cols <> Me.shape.Item1 * Me.shape.Item2 Then
            Throw New System.Exception("Matrix dimensions do not conform for reshape")
        End If
        Dim result As New Matrix(rows, cols)
        Dim i As Integer = 0
        For Each d In val(Me)
            result.item(Math.Truncate(i / cols) + 1, (i Mod cols) + 1) = d
            i += 1
        Next
        Return result
    End Function  'COMPLETED -- Rehapes a matrix in to another matrix with shape = (rows, cols)
    
    Public Sub transposeSelf() Implements Tensor.transposeSelf
        tState = Not tState
        shape = New Tuple(Of Integer, Integer)(Me.shape.Item2, Me.shape.Item1)
    End Sub 'COMPLETED -- Subroutine used to swap "ij" and also swap indices of shape - transposes the current matrix - Does not create an instance

    Public Function clone() As Tensor Implements Tensor.clone
        Dim cloned As New Matrix(Me.shape.Item1, Me.shape.Item2)
        cloned.values = Me.values.Clone
        cloned.tState = Me.tState
        'For a matrix to be cloned, values tState and shape variables are required to be the same
        Return cloned
    End Function 'completed -- function returns an identical matrix (A clone)
    
        Public Shared Function conv(ByVal m As Matrix, ByVal kernel As Matrix, Optional ByVal stridesx As Integer = 1, Optional ByVal stridesy As Integer = 1, Optional ByVal padding As String = "valid") As Matrix
        Dim paddy, paddx As Integer
        If padding = "full" Then
            'For a full convolution, the Matrix should be first zero-padded such that the every element in the matrix can be used to convolve with the kernel, and then a Valid convolution is applied.
            m = Matrix.padd(m, kernel.shape.Item1 - 1, kernel.shape.Item2 - 1)
            Return conv(m, kernel, stridesx, stridesy, "valid")
        End If
        If padding = "same" Then
            If ((m.shape.Item1 Mod stridesy) = 0) Then
                paddy = Math.Max(kernel.shape.Item1 - stridesy, 0)
            Else
                paddy = Math.Max(kernel.shape.Item1 - (m.shape.Item1 Mod stridesy), 0)
            End If
            If ((m.shape.Item2 Mod stridesx) = 0) Then
                paddx = Math.Max(kernel.shape.Item2 - stridesx, 0)
            Else
                paddx = Math.Max(kernel.shape.Item2 - (m.shape.Item2 Mod stridesx), 0)
            End If
            m = Matrix.addcol(m, 1, Math.Floor(paddy / 2), 1)
            m = Matrix.addcol(m, m.shape.Item1 + 1, paddy - Math.Floor(paddy / 2), 1)
            m = Matrix.addcol(m, 1, Math.Floor(paddx / 2), 0)
            m = Matrix.addcol(m, m.shape.Item2 + 1, paddx - Math.Floor(paddx / 2), 0)
            'The amount of padding done for SAME convolution follows the tensorlflow guidlines for the amount of padding
            Return conv(m, kernel, stridesx, stridesy, "valid")
        ElseIf padding = "valid" Then
            Dim result As New Matrix(Math.Truncate((m.shape.Item1 - kernel.getshape(0)) / stridesy) + 1, Math.Truncate((m.shape.Item2 - kernel.getshape(1)) / stridesx) + 1)
            Dim i As Integer = 0
            'The following code is used to compute the resulting convolved Matrix
            'The dot product is used as convolution is essentially a series of dot products
            For Each S In submatrix(m, kernel.shape.Item2, kernel.shape.Item1, stridesx, stridesy)
                result.values(Math.Truncate(i / result.shape.Item2), i Mod result.shape.Item2) = Matrix.dotsum(S, kernel)
                i += 1
            Next
            Return result
        End If
        Console.WriteLine(padding)
        Throw New System.Exception("Padding must be either valid, same or full")
    End Function 'COMPLETED -- Returns the convolution after a kernel have been applied
    
        Public Overloads Shared Function join(ByVal m As Matrix, ByVal n As Matrix, ByVal index As Integer) As Matrix
        If m.shape.Item1 <> n.shape.Item1 Then
            'An error is thrown here is both Matrices do not have the same number of rows
            Throw New System.Exception("Number of Rows must be the same for both Matrices")
        End If
        Dim result As New Matrix(m.shape.Item1, m.shape.Item2 + n.shape.Item2)
        Dim i As Integer = 0
        For k As Integer = 0 To m.shape.Item2 - 1
            If i = index Then
                i += n.shape.Item2
            End If
            For l As Integer = 0 To m.shape.Item1 - 1
                result.values(l, i) = m.item(l + 1, k + 1)
            Next
            i += 1
        Next
        For k As Integer = 0 To n.shape.Item2 - 1
            For l As Integer = 0 To m.shape.Item1 - 1
                result.values(l, index + k) = n.item(l + 1, k + 1)
            Next
        Next
        Return result
    End Function 'COMPLETED -- Concatenates a Matrix(M) to another Matrix(N) at the specified axis

    Public Overloads Shared Function join(ByVal m As Matrix, ByVal n As Matrix, ByVal index As Integer) As Matrix
        If m.shape.Item1 <> n.shape.Item1 Then
            'An error is thrown here is both Matrices do not have the same number of rows
            Throw New System.Exception("Number of Rows must be the same for both Matrices")
        End If
        Dim result As New Matrix(m.shape.Item1, m.shape.Item2 + n.shape.Item2)
        Dim i As Integer = 0
        For k As Integer = 0 To m.shape.Item2 - 1
            If i = index Then
                i += n.shape.Item2
            End If
            For l As Integer = 0 To m.shape.Item1 - 1
                result.values(l, i) = m.item(l + 1, k + 1)
            Next
            i += 1
        Next
        For k As Integer = 0 To n.shape.Item2 - 1
            For l As Integer = 0 To m.shape.Item1 - 1
                result.values(l, index + k) = n.item(l + 1, k + 1)
            Next
        Next
        Return result
    End Function 'COMPLETED -- Concatenates a Matrix(M) to another Matrix(N) at the specified axis
    
    Public Function maxpool(ByVal kernelx As Integer, ByVal kernely As Integer, ByVal stridesx As Integer, ByVal stridesy As Integer) As Matrix
        Dim result As New Matrix(((Me.shape.Item1 - kernely) / stridesy) + 1, ((Me.shape.Item2 - kernelx) / stridesx) + 1)
        Dim i As Integer = 0
        'Following code is used to select the Maximum element out of each submatrix
        For Each m In submatrix(Me, kernelx, kernely, stridesx, stridesy) 'submatrix is a function of type ienumerable
            result.values(Math.Truncate(i / result.shape.Item2), i Mod result.shape.Item2) = m.max()
            i += 1
        Next
        Return result
    End Function 'Completed -- Applies maxpooling to a matrix with a kernel of size = (kernelx, kernely)
    
    Public Shared Function rotate(ByVal m As Matrix, Optional ByVal theta As Integer = 1) As Matrix
        If theta = 0 Then 'If theta is 0, then no rotation is required
            Return m
        Else
            'Rotating a matrix 90 degrees clockwise is the same as transposing the matrix, and then reversing each row
            Dim transposed As Matrix = m.transpose
            Dim result As New Matrix(transposed.shape.Item1, 0)
            For Each col In transposed.columns.Reverse
                result = Matrix.join(result, col, result.shape.Item2)
            Next
            Return rotate(result, theta - 1) 'As theta is a multiple of 90 and rotation is a commutative operation, if theta > 1, then return matrix rotated by theta-1 times
        End If
    End Function 'COMPLETED -- Rotates the matrix by theta * 90
    
    Public Function oneHot(ByVal num_classes As Integer) As Matrix
        'oneHot converts a row vector into a matrix of shape(num_classes, num_samples). Each corresponding item in the row vector is then used
        'to select each column in the resulting matrix by placing a 1, whilst the rest of the items are set to 0.
        If Me.getshape(0) <> 1 Then 'Checking to see if the matrix is a row vector or not
            Throw New System.Exception("Matrix must be a row vector for one hot")
        End If
        Dim oneHotArr(num_classes - 1, Me.getshape(1) - 1) As Double
        For j As Integer = 0 To Me.getshape(1) - 1
            oneHotArr(Me.item(1, j + 1), j) = 1 'Useing the items in the matrix to select, place a 1, in the resultant matrix
        Next
        Return New Matrix(oneHotArr)
    End Function 'COMPLETED -- Returns the onehot of a matri
    
    Public Function invOneHot() As Matrix
        Dim result As New Matrix(1, Me.getshape(1))
        For j As Integer = 1 To Me.getshape(1)
            Dim maxval As Double = Double.MinValue 'Variable used to store the maximum item in this column vector
            Dim pos As Integer = 0 'Variable used to store the position of the maximum item in the vector
            For i As Integer = 1 To Me.getshape(0)
                If Me.item(i, j) > maxval Then 'If the maximum item is greater than assign the maxmav to that item and "pos" to the position of that maxmimum value
                    maxval = Me.item(i, j)
                    pos = i
                End If
            Next
            result.item(1, j) = pos - 1
        Next
        Return result
    End Function 'COMPLETED -- Returns the inverse of Onehot encoding to a matrix
    
    Public Function normalize(Optional ByVal mean As Double = 0, Optional ByVal std As Double = 1) As Tensor Implements Tensor.normalize
        Dim means As Matrix = Matrix.sum(Me, 0) / Me.getshape(0) 'Finds the column means
        Dim stds As Matrix = Matrix.op(AddressOf Math.Pow, (Matrix.sum(Matrix.op(AddressOf Math.Pow, Me, 2), 0) - Matrix.op(AddressOf Math.Pow, Me, 2) * Me.getshape(0)) / (Me.getshape(0) - 1), 0.5) 'Finds the std for this particular colum

        Dim result As New Matrix(Me.getshape(0), Me.getshape(1))
        For i As Integer = 1 To Me.getshape(0)
            For j As Integer = 1 To Me.getshape(1)
                result.item(i, j) = (Me.item(i, j) - means.item(1, j)) / stds.item(1, j) 'normalises each item in the matrix
            Next
        Next
        Return (result * std) + mean
    End Function  'COMPLETED -- Returns the normalised version of a particular matrix
    
    Public Shared Function matmul(ByVal x As Matrix, ByVal y As Matrix) As Matrix
        If x.shape.Item2 <> y.shape.Item1 Then
            Console.WriteLine("Shape of A is {0}, {1}. Shape of B is {2}, {3}", x.shape.Item1, x.shape.Item2, y.shape.Item1, y.shape.Item2)
            Throw New System.Exception("Shapes do not conform for matrix multiplication")
        End If
        'The following code, is for matrix multiplication using the standard way
        Dim result As New Matrix(x.shape.Item1, y.shape.Item2)
        For i As Integer = 0 To x.shape.Item1 - 1
            For j As Integer = 0 To y.shape.Item2 - 1
                Dim sum As Decimal = 0
                For k = 0 To x.shape.Item2 - 1
                    Try
                        sum += x.item(i + 1, k + 1) * y.item(k + 1, j + 1)
                    Catch ex As Exception
                        Throw New System.Exception("Value was too large or too Small.")
                    End Try
                Next
                result.values(i, j) = sum
            Next
        Next
        Return result
    End Function  'COMPLETED -- Returns the product of Matrix Multiplication
    
    Public Shared Function dotsum(ByVal x As Matrix, ByVal y As Matrix) As Boolean
        Dim product As Matrix = y * x
        Return Matrix.sum(Matrix.sum(product, 1)).item(1, 1)
    End Function 'COMPLETED -- Returns the sum of all the elementwise multiplication of the 2 matrices
    
    Public Shared Operator =(ByVal a As Matrix, ByVal b As Matrix) As Boolean
        If Not a Like b Then
            Return False
        End If
        For i As Integer = 1 To a.shape.Item1
            For j As Integer = 1 To a.shape.Item2
                If a.item(i, j) <> b.item(i, j) Then 'Statement compare each value in the matrix
                    Return False
                End If
            Next
        Next
        Return True
    End Operator 'Completed -- Checks if all items in the matrix are the same
    
    Public Overloads Iterator Function columns() As IEnumerable(Of Tensor)
        For j As Integer = 1 To Me.shape.Item2
            Dim result As New Matrix(Me.shape.Item1, 1)
            For i As Integer = 1 To Me.shape.Item1
                result.values(i - 1, 0) = Me.item(i, j) 'Copies items of the column of the matrix to another matrix
            Next
            Yield result
        Next
    End Function 'COMPLETED -- Returns an enumerable of columns of the matrix

    Public Shared Iterator Function submatrix(ByVal m As Matrix, ByVal kernelx As Integer, ByVal kernely As Integer, ByVal stridesx As Integer, ByVal stridesy As Integer) As IEnumerable(Of Matrix)
        For i As Integer = 1 To m.getshape(0) - kernely + 1 Step stridesy
            For j As Integer = 1 To m.getshape(1) - kernelx + 1 Step stridesx
                Yield m.item(i, i + kernely - 1, j, j + kernelx - 1)
            Next
        Next
    End Function 'COMPLETED -- Returns a collection of matrices, by sliding a window of length (kernelx, kernely) and using strides = (stridesx, stridesy)

    Public Shared Iterator Function val(ByVal m As Matrix, Optional stepx As Integer = 1, Optional stepy As Integer = 1) As IEnumerable(Of Double)
        For i As Integer = 1 To m.getshape(0) Step stepy
            For j As Integer = 1 To m.getshape(1) Step stepx
                Yield m.item(i, j)
            Next
        Next
    End Function 'COMPLETED -- Returns items from a matrix, using a stepsize of (stepx, stepy)
    
\end{minted}

Below is the code for some of the volume operations that are being used in my project: 
\begin{minted}[
frame=lines,
breaklines=true
]{vb.net}

    Public Property split(ByVal i_start As Integer, ByVal i_end As Integer, ByVal j_start As Integer, ByVal j_end As Integer, ByVal k As Integer) As Matrix
        Get
            Dim result As New Matrix(i_end - i_start + 1, j_end - j_start + 1)
            For i As Integer = i_start To i_end
                For j As Integer = j_start To j_end
                    result.item(i - i_start + 1, j - j_start + 1) = Me.item(i, j, k)
                Next
            Next
            Return result
        End Get
        Set(ByVal value As Matrix)
            For i As Integer = i_start To i_end
                For j As Integer = j_start To j_end
                    Me.item(i, j, k) = value.item(i - i_start + 1, j - j_start + 1)
                Next
            Next
        End Set
    End Property 'COMPLETED -- Property used to set/select a portion of a volume
    
    Public Function rotate(ByVal theta As Integer) As Volume
        If theta = 0 Then
            Return Me 'If theta is 0, then return identity
        Else
            Return op(AddressOf Matrix.rotate, Me).rotate(theta - 1) 'Else rotate each layer in the Volume, and then rotate by (theta-1)*90 degrees
        End If
    End Function 'COMPLETED -- Function rotates the Volume by theta * 90 degree
    
    Public Function normalize(Optional ByVal mean As Double = 0, Optional ByVal std As Double = 1) As Tensor Implements Tensor.normalize
        Dim n As Integer = Me.shape.Item1 * Me.shape.Item2 * Me.values.Count
        Dim means As Double = Me.values.Select(Function(x) Matrix.sum(x).item(1, 1)).Sum / n
        Dim stds As Double = Math.Sqrt((Me.values.Select(Function(x) Matrix.sum(x * x).item(1, 1)).Sum - (mean * mean * n)) / (n - 1))
        Dim result As New List(Of Matrix)
        For Each M In Volume.Items(Me)
            result.Add((M - means) / stds)
        Next
        Return New Volume(result)
    End Function  'COMPLETED -- Returns a volume whose layers are normlised using all the elements in the volume
    
    Public Function mean(ByVal axis As Integer) As Matrix
        Dim result As New Matrix(Me.shape.Item1, Me.shape.Item2)
        If axis = 2 Then
            For Each M In Me.values
                result += M
            Next
            Return result / Me.values.Count
        Else
            Throw New System.Exception("axis 1 or 2 has not yet been implemented yet for Volume")
        End If
    End Function 'COMPLETED -- Returns the mean of a volume in a specified dimension. Currently, only works for axis = 2
    
    Public Function transpose() As Volume
        Return op(AddressOf Matrix.transpose, Me)
    End Function 'COMPLETED -- Function returns the transpose of each element in the matrix
    
    Public Function clone() As Tensor Implements Tensor.clone
        Dim cloned As New List(Of Matrix)
        For Each m In Items(Me)
            cloned.Add(m)
        Next
        Return New Volume(cloned)
    End Function 'COMPLETED -- Function returns a clone of the current volume
    
    Public Shared Function conv2d(ByVal v As Volume, ByVal kernels As Volume, Optional stridesx As Integer = 1, Optional stridesy As Integer = 1, Optional padding As String = "valid") As Volume
        'conv2d is applying a convolution using in 2 dimensions. This means that every 2d kernel is applied to every layer in the volume.
        Dim result_values As New List(Of Matrix) : Dim all_channels As List(Of Matrix) = Items(v).ToList

        For Each k In Items(kernels)
            Dim temp As Matrix = Matrix.conv(all_channels(0), k, stridesx, stridesy, padding)
            For Each M In all_channels.GetRange(1, all_channels.Count - 1)
                temp += Matrix.conv(M, k, stridesx, stridesy, padding) 'Summing up the result of all the convolutions for that particular kernel
            Next
            result_values.Add(temp)
        Next
        Return New Volume(result_values)
    End Function 'COMPLETED -- Applies 2d convolutio
    
    Public Shared Function maxpool(ByVal filter As Volume, ByVal kernely As Integer, ByVal kernelx As Integer, ByVal stridesy As Integer, ByVal stridesx As Integer) As Volume
        Dim result As New List(Of Matrix)
        For Each M In Items(filter)
            result.Add(M.maxpool(kernelx, kernely, stridesx, stridesy))
        Next
        Return New Volume(result)
    End Function 'COMPLETED -- Returns the maxpooling of a volume using a kernel of shape (kernelx, kernely) and step size = (stridesx, stridesy)
    
    Public Shared Function op(ByVal x As Volume, ByVal f As Func(Of Matrix, Matrix, Matrix), ByVal y As Volume) As Volume
        Dim result As New List(Of Matrix)
        For i As Integer = 0 To x.values.Count - 1
            result.Add(f.Invoke(x.values(i), y.values(i)))
        Next
        Return New Volume(result)
    End Function 'COMPLETED -- Applies a function f(Matrix, Matrix) -> Matrix, to all the layers in the Volumes x and y

    Public Shared Function op(ByVal v As Volume, ByVal f As Func(Of Matrix, Matrix, Matrix), ByVal m As Matrix) As Volume
        Dim result As New Volume(v.shape.Item1, v.shape.Item2, 0)
        For Each x In Items(v)
            result.values.Add(f.Invoke(x, m))
        Next
        Return result
    End Function 'COMPLETED -- Function applies a function "f" to a Volume "x" and "y" layer-wis
    
    Public Overloads Iterator Function Items() As IEnumerable(Of Tensor)
        For Each Matrix In Me.values
            Yield Matrix
        Next
    End Function 'COMPLETED -- Returns an IEnumerable of all the layers in the Volume

    Public Shared Iterator Function subvolume(ByVal v As Volume, ByVal kernelx As Integer, ByVal kernely As Integer, ByVal stridesx As Integer, ByVal stridesy As Integer) As IEnumerable(Of Volume)
        For i As Integer = 1 To v.shape.Item1 - kernely + 1 Step stridesy
            For j As Integer = 1 To v.shape.Item2 - kernelx + 1 Step stridesx
                Yield v.split(i, i + kernely - 1, j, j + kernelx - 1)
            Next
        Next
    End Function 'COMPLETED -- Returns a collection of volumes, by sliding a window of length (kernelx, kernely) and using strides = (stridesx, stridesy) on each individual layer
   
    Public Shared Function cast(ByVal matrixList As List(Of Matrix), ByVal rows As Integer, ByVal cols As Integer) As List(Of Volume)
        Dim l_v As New List(Of Volume)
        For Each M In matrixList
            l_v.Add(Volume.cast(M, rows, cols))
        Next
        Return l_v
    End Function 'COMPLETED -- Casts a list of matrices into volumes elementwise

    Public Shared Function cast(ByVal v As Volume, ByVal rows As Integer, ByVal cols As Integer) As Matrix
        If rows * cols <> v.shape.Item1 * v.shape.Item2 * v.values.Count Then
            Throw New System.Exception("Dimensions for matrix must only be sufficient to store all the items in the Volume")
        End If
        Dim result As New Matrix(rows, cols)
        Dim i As Integer = 0
        For Each M In Volume.Items(v)
            For Each k In Matrix.val(M)
                result.item((Math.Truncate(i / result.getshape(1)) + 1), (i Mod result.getshape(1)) + 1) = k 'Assigning each element in result with 
                'its corresponding value in the Volume "v"
                i += 1
            Next
        Next
        Return result
    End Function 'COMPLETED -- Function used to cast a Volume into a matrix of shape = (rows, cols)

    Public Shared Function cast(ByVal m As Matrix, ByVal rows As Integer, ByVal cols As Integer) As Volume
        Dim result As New Volume(rows, cols, m.getshape(0) * m.getshape(1) / (rows * cols))
        Dim i As Integer = 0
        For Each d In Matrix.val(m)
            result.item(((Math.Truncate(i / (cols))) Mod rows) + 1, (i Mod (cols)) + 1, Math.Truncate(i / ((cols * rows)))) = d
            i += 1
        Next
        Return result
    End Function 'COMPLETED -- Function casts a matrix into a volume of shape = (rows, cols)

\end{minted}