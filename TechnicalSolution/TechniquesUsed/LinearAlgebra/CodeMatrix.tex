Below is the code for some of the matrix operations that are being used in my project: 
\begin{minted}[
frame=lines,
breaklines=true
]{vb.net}
    Public Function reshape(ByVal rows As Integer, ByVal cols As Integer) As Matrix
        If rows * cols <> Me.shape.Item1 * Me.shape.Item2 Then
            Throw New System.Exception("Matrix dimensions do not conform for reshape")
        End If
        Dim result As New Matrix(rows, cols)
        Dim i As Integer = 0
        For Each d In val(Me)
            result.item(Math.Truncate(i / cols) + 1, (i Mod cols) + 1) = d
            i += 1
        Next
        Return result
    End Function  'COMPLETED -- Rehapes a matrix in to another matrix with shape = (rows, cols)
    
    Public Sub transposeSelf() Implements Tensor.transposeSelf
        tState = Not tState
        shape = New Tuple(Of Integer, Integer)(Me.shape.Item2, Me.shape.Item1)
    End Sub 'COMPLETED -- Subroutine used to swap "ij" and also swap indices of shape - transposes the current matrix - Does not create an instance

    Public Function clone() As Tensor Implements Tensor.clone
        Dim cloned As New Matrix(Me.shape.Item1, Me.shape.Item2)
        cloned.values = Me.values.Clone
        cloned.tState = Me.tState
        'For a matrix to be cloned, values tState and shape variables are required to be the same
        Return cloned
    End Function 'completed -- function returns an identical matrix (A clone)
    
        Public Shared Function conv(ByVal m As Matrix, ByVal kernel As Matrix, Optional ByVal stridesx As Integer = 1, Optional ByVal stridesy As Integer = 1, Optional ByVal padding As String = "valid") As Matrix
        Dim paddy, paddx As Integer
        If padding = "full" Then
            'For a full convolution, the Matrix should be first zero-padded such that the every element in the matrix can be used to convolve with the kernel, and then a Valid convolution is applied.
            m = Matrix.padd(m, kernel.shape.Item1 - 1, kernel.shape.Item2 - 1)
            Return conv(m, kernel, stridesx, stridesy, "valid")
        End If
        If padding = "same" Then
            If ((m.shape.Item1 Mod stridesy) = 0) Then
                paddy = Math.Max(kernel.shape.Item1 - stridesy, 0)
            Else
                paddy = Math.Max(kernel.shape.Item1 - (m.shape.Item1 Mod stridesy), 0)
            End If
            If ((m.shape.Item2 Mod stridesx) = 0) Then
                paddx = Math.Max(kernel.shape.Item2 - stridesx, 0)
            Else
                paddx = Math.Max(kernel.shape.Item2 - (m.shape.Item2 Mod stridesx), 0)
            End If
            m = Matrix.addcol(m, 1, Math.Floor(paddy / 2), 1)
            m = Matrix.addcol(m, m.shape.Item1 + 1, paddy - Math.Floor(paddy / 2), 1)
            m = Matrix.addcol(m, 1, Math.Floor(paddx / 2), 0)
            m = Matrix.addcol(m, m.shape.Item2 + 1, paddx - Math.Floor(paddx / 2), 0)
            'The amount of padding done for SAME convolution follows the tensorlflow guidlines for the amount of padding
            Return conv(m, kernel, stridesx, stridesy, "valid")
        ElseIf padding = "valid" Then
            Dim result As New Matrix(Math.Truncate((m.shape.Item1 - kernel.getshape(0)) / stridesy) + 1, Math.Truncate((m.shape.Item2 - kernel.getshape(1)) / stridesx) + 1)
            Dim i As Integer = 0
            'The following code is used to compute the resulting convolved Matrix
            'The dot product is used as convolution is essentially a series of dot products
            For Each S In submatrix(m, kernel.shape.Item2, kernel.shape.Item1, stridesx, stridesy)
                result.values(Math.Truncate(i / result.shape.Item2), i Mod result.shape.Item2) = Matrix.dotsum(S, kernel)
                i += 1
            Next
            Return result
        End If
        Console.WriteLine(padding)
        Throw New System.Exception("Padding must be either valid, same or full")
    End Function 'COMPLETED -- Returns the convolution after a kernel have been applied
    
        Public Overloads Shared Function join(ByVal m As Matrix, ByVal n As Matrix, ByVal index As Integer) As Matrix
        If m.shape.Item1 <> n.shape.Item1 Then
            'An error is thrown here is both Matrices do not have the same number of rows
            Throw New System.Exception("Number of Rows must be the same for both Matrices")
        End If
        Dim result As New Matrix(m.shape.Item1, m.shape.Item2 + n.shape.Item2)
        Dim i As Integer = 0
        For k As Integer = 0 To m.shape.Item2 - 1
            If i = index Then
                i += n.shape.Item2
            End If
            For l As Integer = 0 To m.shape.Item1 - 1
                result.values(l, i) = m.item(l + 1, k + 1)
            Next
            i += 1
        Next
        For k As Integer = 0 To n.shape.Item2 - 1
            For l As Integer = 0 To m.shape.Item1 - 1
                result.values(l, index + k) = n.item(l + 1, k + 1)
            Next
        Next
        Return result
    End Function 'COMPLETED -- Concatenates a Matrix(M) to another Matrix(N) at the specified axis

    Public Overloads Shared Function join(ByVal m As Matrix, ByVal n As Matrix, ByVal index As Integer) As Matrix
        If m.shape.Item1 <> n.shape.Item1 Then
            'An error is thrown here is both Matrices do not have the same number of rows
            Throw New System.Exception("Number of Rows must be the same for both Matrices")
        End If
        Dim result As New Matrix(m.shape.Item1, m.shape.Item2 + n.shape.Item2)
        Dim i As Integer = 0
        For k As Integer = 0 To m.shape.Item2 - 1
            If i = index Then
                i += n.shape.Item2
            End If
            For l As Integer = 0 To m.shape.Item1 - 1
                result.values(l, i) = m.item(l + 1, k + 1)
            Next
            i += 1
        Next
        For k As Integer = 0 To n.shape.Item2 - 1
            For l As Integer = 0 To m.shape.Item1 - 1
                result.values(l, index + k) = n.item(l + 1, k + 1)
            Next
        Next
        Return result
    End Function 'COMPLETED -- Concatenates a Matrix(M) to another Matrix(N) at the specified axis
    
    Public Function maxpool(ByVal kernelx As Integer, ByVal kernely As Integer, ByVal stridesx As Integer, ByVal stridesy As Integer) As Matrix
        Dim result As New Matrix(((Me.shape.Item1 - kernely) / stridesy) + 1, ((Me.shape.Item2 - kernelx) / stridesx) + 1)
        Dim i As Integer = 0
        'Following code is used to select the Maximum element out of each submatrix
        For Each m In submatrix(Me, kernelx, kernely, stridesx, stridesy) 'submatrix is a function of type ienumerable
            result.values(Math.Truncate(i / result.shape.Item2), i Mod result.shape.Item2) = m.max()
            i += 1
        Next
        Return result
    End Function 'Completed -- Applies maxpooling to a matrix with a kernel of size = (kernelx, kernely)
    
    Public Shared Function rotate(ByVal m As Matrix, Optional ByVal theta As Integer = 1) As Matrix
        If theta = 0 Then 'If theta is 0, then no rotation is required
            Return m
        Else
            'Rotating a matrix 90 degrees clockwise is the same as transposing the matrix, and then reversing each row
            Dim transposed As Matrix = m.transpose
            Dim result As New Matrix(transposed.shape.Item1, 0)
            For Each col In transposed.columns.Reverse
                result = Matrix.join(result, col, result.shape.Item2)
            Next
            Return rotate(result, theta - 1) 'As theta is a multiple of 90 and rotation is a commutative operation, if theta > 1, then return matrix rotated by theta-1 times
        End If
    End Function 'COMPLETED -- Rotates the matrix by theta * 90
    
    Public Function oneHot(ByVal num_classes As Integer) As Matrix
        'oneHot converts a row vector into a matrix of shape(num_classes, num_samples). Each corresponding item in the row vector is then used
        'to select each column in the resulting matrix by placing a 1, whilst the rest of the items are set to 0.
        If Me.getshape(0) <> 1 Then 'Checking to see if the matrix is a row vector or not
            Throw New System.Exception("Matrix must be a row vector for one hot")
        End If
        Dim oneHotArr(num_classes - 1, Me.getshape(1) - 1) As Double
        For j As Integer = 0 To Me.getshape(1) - 1
            oneHotArr(Me.item(1, j + 1), j) = 1 'Useing the items in the matrix to select, place a 1, in the resultant matrix
        Next
        Return New Matrix(oneHotArr)
    End Function 'COMPLETED -- Returns the onehot of a matri
    
    Public Function invOneHot() As Matrix
        Dim result As New Matrix(1, Me.getshape(1))
        For j As Integer = 1 To Me.getshape(1)
            Dim maxval As Double = Double.MinValue 'Variable used to store the maximum item in this column vector
            Dim pos As Integer = 0 'Variable used to store the position of the maximum item in the vector
            For i As Integer = 1 To Me.getshape(0)
                If Me.item(i, j) > maxval Then 'If the maximum item is greater than assign the maxmav to that item and "pos" to the position of that maxmimum value
                    maxval = Me.item(i, j)
                    pos = i
                End If
            Next
            result.item(1, j) = pos - 1
        Next
        Return result
    End Function 'COMPLETED -- Returns the inverse of Onehot encoding to a matrix
    
    Public Function normalize(Optional ByVal mean As Double = 0, Optional ByVal std As Double = 1) As Tensor Implements Tensor.normalize
        Dim means As Matrix = Matrix.sum(Me, 0) / Me.getshape(0) 'Finds the column means
        Dim stds As Matrix = Matrix.op(AddressOf Math.Pow, (Matrix.sum(Matrix.op(AddressOf Math.Pow, Me, 2), 0) - Matrix.op(AddressOf Math.Pow, Me, 2) * Me.getshape(0)) / (Me.getshape(0) - 1), 0.5) 'Finds the std for this particular colum

        Dim result As New Matrix(Me.getshape(0), Me.getshape(1))
        For i As Integer = 1 To Me.getshape(0)
            For j As Integer = 1 To Me.getshape(1)
                result.item(i, j) = (Me.item(i, j) - means.item(1, j)) / stds.item(1, j) 'normalises each item in the matrix
            Next
        Next
        Return (result * std) + mean
    End Function  'COMPLETED -- Returns the normalised version of a particular matrix
    
    Public Shared Function matmul(ByVal x As Matrix, ByVal y As Matrix) As Matrix
        If x.shape.Item2 <> y.shape.Item1 Then
            Console.WriteLine("Shape of A is {0}, {1}. Shape of B is {2}, {3}", x.shape.Item1, x.shape.Item2, y.shape.Item1, y.shape.Item2)
            Throw New System.Exception("Shapes do not conform for matrix multiplication")
        End If
        'The following code, is for matrix multiplication using the standard way
        Dim result As New Matrix(x.shape.Item1, y.shape.Item2)
        For i As Integer = 0 To x.shape.Item1 - 1
            For j As Integer = 0 To y.shape.Item2 - 1
                Dim sum As Decimal = 0
                For k = 0 To x.shape.Item2 - 1
                    Try
                        sum += x.item(i + 1, k + 1) * y.item(k + 1, j + 1)
                    Catch ex As Exception
                        Throw New System.Exception("Value was too large or too Small.")
                    End Try
                Next
                result.values(i, j) = sum
            Next
        Next
        Return result
    End Function  'COMPLETED -- Returns the product of Matrix Multiplication
    
    Public Shared Function dotsum(ByVal x As Matrix, ByVal y As Matrix) As Boolean
        Dim product As Matrix = y * x
        Return Matrix.sum(Matrix.sum(product, 1)).item(1, 1)
    End Function 'COMPLETED -- Returns the sum of all the elementwise multiplication of the 2 matrices
    
    Public Shared Operator =(ByVal a As Matrix, ByVal b As Matrix) As Boolean
        If Not a Like b Then
            Return False
        End If
        For i As Integer = 1 To a.shape.Item1
            For j As Integer = 1 To a.shape.Item2
                If a.item(i, j) <> b.item(i, j) Then 'Statement compare each value in the matrix
                    Return False
                End If
            Next
        Next
        Return True
    End Operator 'Completed -- Checks if all items in the matrix are the same
    
    Public Overloads Iterator Function columns() As IEnumerable(Of Tensor)
        For j As Integer = 1 To Me.shape.Item2
            Dim result As New Matrix(Me.shape.Item1, 1)
            For i As Integer = 1 To Me.shape.Item1
                result.values(i - 1, 0) = Me.item(i, j) 'Copies items of the column of the matrix to another matrix
            Next
            Yield result
        Next
    End Function 'COMPLETED -- Returns an enumerable of columns of the matrix

    Public Shared Iterator Function submatrix(ByVal m As Matrix, ByVal kernelx As Integer, ByVal kernely As Integer, ByVal stridesx As Integer, ByVal stridesy As Integer) As IEnumerable(Of Matrix)
        For i As Integer = 1 To m.getshape(0) - kernely + 1 Step stridesy
            For j As Integer = 1 To m.getshape(1) - kernelx + 1 Step stridesx
                Yield m.item(i, i + kernely - 1, j, j + kernelx - 1)
            Next
        Next
    End Function 'COMPLETED -- Returns a collection of matrices, by sliding a window of length (kernelx, kernely) and using strides = (stridesx, stridesy)

    Public Shared Iterator Function val(ByVal m As Matrix, Optional stepx As Integer = 1, Optional stepy As Integer = 1) As IEnumerable(Of Double)
        For i As Integer = 1 To m.getshape(0) Step stepy
            For j As Integer = 1 To m.getshape(1) Step stepx
                Yield m.item(i, j)
            Next
        Next
    End Function 'COMPLETED -- Returns items from a matrix, using a stepsize of (stepx, stepy)
    
\end{minted}
